% В данном файле предлагается располагать пользовательские команды

% Удобная команда для вставки картинок
\newcommand{\img}[5][htb]{
  	\begin{figure}[#1]
    		\centering
    		\includegraphics[width = #5\linewidth]{./img/#3}
   	 	\caption{#4}
	 	\label{#2}
  	\end{figure}
}

% Включение сквозной нумерации рисунков
\counterwithout{figure}{chapter}

% Изменение команды из шаблона bmstu для корректной вставки номера страницы
% списка литературы
\renewcommand{\makebibliography}
{	
\printbibliography[heading=bibintoc,title={СПИСОК ИСПОЛЬЗОВАННЫХ ИСТОЧНИКОВ}]
}

% Команда для добавления сокращения
\newcommand{\abbr}[2]{
	\DeclareAcronym{#1}{
    	short={#1},
    	long={--- #2},
    }
}

% Команда для вставки списка сокращений. По умолчанию выводится сортированный
% список сокращений. Если необходимо вывести список в порядке указания (в файле
% abbreviations.tex), то необходимо указать параметр sort=false.
\newcommand{\printabbr}{
	\printacronyms[display=all,sort=true,name={ОБОЗНАЧЕНИЯ И СОКРАЩЕНИЯ},
	preamble={В	настоящей расчетно-пояснительной записке применяют следующие 
	сокращения и обозначения.}] 
	\addcontentsline{toc}{chapter}{ОБОЗНАЧЕНИЯ И СОКРАЩЕНИЯ}
}